
\documentclass{article}
\usepackage{amsmath}
\usepackage{caption}
\usepackage{placeins}
\usepackage{graphicx}
\usepackage{subcaption}
\usepackage{setspace}
%\usepackage[active,tightpage]{preview}
\usepackage{natbib}
\bibpunct{(}{)}{,}{a}{}{;} 
\usepackage{url}
\usepackage{nth}
\usepackage{authblk}
% for the d in integrals
\newcommand{\dd}{\; \mathrm{d}}
\newcommand{\tc}{\quad\quad\text{,}}
\newcommand{\tp}{\quad\quad\text{.}}
\title{Mexico!}

\author[1]{JM ABurto\thanks{email here}}
\author[2]{Tim Riffe}
\affil[1]{Max Planck/EDSD}
\affil[2]{Max Planck Institute for Demographic Research}


\begin{document}
\maketitle

\section{Background}
What is the question we are asking? Has anyone asked it before? What did they do
and find? What are we going to do that's different and what will we find?

\section{Data \& Methods} 

Cause of death classification and age reporting are considered to be inaccurate in death registration in older ages. For this reason, we truncate analysis at age 75. This is also a reasonable decision because most changes in life expectancy are likely due to changes in mortality patterns below the age of 75. There are different ways to summarize mortality conditions in specified age ranges, and we choose to focus on temporary life expectancy between the ages of 0 and 75 (Arriaga, 1982), at times breaking this quantity into large age groups.

Exposures came from 

Death counts were tabulated by the authors based on official microdata available from 
\subsection{Classification of causes of death}
The period of years covered in this study spans both ICD 9 and ICD 10, \ldots find
language\ldots based on XXX (2013) Causes were grouped into X categories based 

Data quality and limitation (appendix?)
In the case of Mexico, data for old ages are not sufficiently reliable at the
state level to perform calculations as specified in the above equations, and so
we truncate anayses at age 75 using the formulas recommended by Arriaga (1982).
\subsection{Best practices lifetable}
This method was first proposed by Vallin \& Mesle (2008), and we summarize it
briefly here. The imaginary best practices lifetable for Mexico is a composite
of the lowest observed mortality rates by age, cause, and state for a given sex
and year. In continous terms, we define life expecancy, $e(0)$, as:
\begin{equation}
\int _0 ^\infty l(x) \dd x \tc
\end{equation}
where $l(x)$ is the survivorship function defined with radix of one, or as a
function of the force of mortality, $\mu(x)$ as:
\begin{equation}
\label{eq:lx}
l(x) = e^{-\int_0^x \mu(a) \dd a}
\end{equation}

In general, $\mu(x)$ can be treated as the sum of $C$ cause-specific mortality
rates at age $x$:
\begin{equation}
\label{eq:mx}
\mu(x) = \sum_1^C \mu_c(x)
\end{equation}

In the case of best practices mortality, and assuming that causes of death are
independent of one another VCR (XXXX), we treat $\mu(x)$ as a composite of the
lowest observed cause-specific mortality rates our of all 32 Mexican states in the given age $x$.

\begin{equation}
\label{eq:mxmin}
\mu(x)^{\star} = \sum_1^C min(\mu_c(x))
\end{equation}

This best practices $\mu^{\star}$ has a unique age profile, and it uniquely
determines a pattern of $l(x)^{\star}$, per \eqref{eq:lx}, that corresponds with a
best practices life expectancy, $e(0)^{\star}$. $e(0)^{\star}$ can be treated as a
sort of maximum presently acheivable life expectancy given the best available
practices and technologies within a given set of populations and assuming
perfect diffussion. It is an imaginary quantity because no particular population
ever acheives this mortality pattern, and none may ever. However, this value is
a real referent not based on a projection of improvements into the future, and
so it bounds our optimism in a practical way.

\subsection{temporary life expectancy}

Given the data, we require an estimate of temporary life expectancy between ages
0 and 75, or the average years of life lived between the ages of 0 and 75 according to a
given set of period mortality rates. We denote this quantity as
${}_{75}e_0$, and its best practices minimum as ${}_{75}e_0^{\star}$. Defined in
terms of $l(x)^\star$

\begin{equation}
{}_{75}e_0^{\star} = \int _0^{75} l(x)^\star \dd x \quad \tc
\end{equation}
or in general for bounded ages, with a lower bound $lb$ and upper bound, $ub$

\begin{equation}
{}_{ub}e_{lb}^{\star} = \frac{\int _{lb}^{ub} l(x)^\star \dd x }{l(lb)^\star}
\quad \tc
\end{equation}




\subsection{decomposing differences in temp life expectancy}

\section{Trends in best practices life expectancy}
The spaghetti plots\ldots

\section{State trends in departures from best practices temp e(0)}
Small multiples maps (time series of maps)

\section{Age and cause contributions to differences}

This will show a few small multiples figures, tbd 

\section{Discussion}
Talk about the role of homicide and other major causes. How many years of life
were lost? (not just expectancy). maybe..

\section{Conclusions}
Homicide is a public health concern..

\end{document}



